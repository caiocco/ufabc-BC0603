\documentclass[a4paper]{article}
\usepackage{amsmath}
\usepackage{tikz}
\usepackage{epigraph}
\usepackage{lipsum}
\usepackage[brazil]{babel}
\usepackage[utf8]{inputenc}
\usepackage[T1]{fontenc}
\usepackage{lmodern}

\renewcommand\epigraphflush{flushright}
\renewcommand\epigraphsize{\normalsize}
\setlength\epigraphwidth{0.7\textwidth}

%\definecolor{titlepagecolor}{cmyk}{1,.60,0,.40}
\definecolor{titlepagecolor}{cmyk}{0,.9,.9,.4}

% \DeclareFixedFont {<cmd>} {<ENC>} {<family>} {<series>} {<shape>} {<size>} 
% http://tex.loria.fr/ctan-doc/macros/latex/doc/html/fntguide/node11.html
\DeclareFixedFont{\titlefont}{T1}{ppl}{b}{it}{0.5in}

\makeatletter                       
\def\printauthor{%                  
    {\large \@author}}              
\makeatother
\author{%
    \textbf{Caio C\'{e}sar Carvalho Ortega} \\
	Registro Acad\^{e}mico 21038515 \\
    Centro de Engenharia, Modelagem e Ci\^{e}ncias Sociais Aplicadas \\
    \texttt{caio.ortega@aluno.ufabc.edu.br}\vspace{20pt} \\
%    Author 2 name \\
%    Department name \\
%    \texttt{email2@example.com}
    }

% The following code is borrowed from: http://tex.stackexchange.com/a/86310/10898

\newcommand\titlepagedecoration{%
\begin{tikzpicture}[remember picture,overlay,shorten >= -10pt]

\coordinate (aux1) at ([yshift=-15pt]current page.north east);
\coordinate (aux2) at ([yshift=-410pt]current page.north east);
\coordinate (aux3) at ([xshift=-4.5cm]current page.north east);
\coordinate (aux4) at ([yshift=-150pt]current page.north east);

\begin{scope}[titlepagecolor!40,line width=12pt,rounded corners=12pt]
\draw
  (aux1) -- coordinate (a)
  ++(225:5) --
  ++(-45:5.1) coordinate (b);
\draw[shorten <= -10pt]
  (aux3) --
  (a) --
  (aux1);
\draw[opacity=0.6,titlepagecolor,shorten <= -10pt]
  (b) --
  ++(225:2.2) --
  ++(-45:2.2);
\end{scope}
\draw[titlepagecolor,line width=8pt,rounded corners=8pt,shorten <= -10pt]
  (aux4) --
  ++(225:0.8) --
  ++(-45:0.8);
\begin{scope}[titlepagecolor!70,line width=6pt,rounded corners=8pt]
\draw[shorten <= -10pt]
  (aux2) --
  ++(225:3) coordinate[pos=0.45] (c) --
  ++(-45:3.1);
\draw
  (aux2) --
  (c) --
  ++(135:2.5) --
  ++(45:2.5) --
  ++(-45:2.5) coordinate[pos=0.3] (d);   
\draw 
  (d) -- +(45:1);
\end{scope}
\end{tikzpicture}%
}

\begin{document}
\begin{titlepage}

\noindent
\titlefont Pensamento cr\'{\i}tico em\par
\titlefont meio \`{a}s chamas\par
\epigraph{(\dots) Um livro \'{e} uma arma carregada na casa vizinha. Queime-o. Descarregue a arma. Fa\c{c}amos uma brecha no esp\'{\i}rito do homem. Quem sabe quem poderia ser alvo do homem lido? Eu? Eu n\~{a}o tenho estômago para eles, nem por um minuto. E assim, quando as casas finalmente se tornaram \`{a} prova de fogo, no mundo inteiro \textemdash voc\^{e} estava certo em sua suposição na noite passada \textemdash, j\'{a} n\~{a}o havia mais necessidade de bombeiros para os velhos fins. Eles receberam uma nova miss\~{a}o, a guarda de nossa paz de esp\'{i}rito, a elimina\caption{c}\~{a}o do nosso compreens\'{i}vel e leg\'{i}timo sentimento de inferioridade: censores, ju\'{i}zes e carrascos oficiais. Eis o nosso papel, Montag, o seu e o meu.}%
{\textit{Beatty em Fahrenheit 451}\\ \textsc{Ray Bradbury}}
\null\vfill
\vspace*{1cm}
\noindent
\hfill
\begin{minipage}{0.60\linewidth}
    \begin{flushright}
        \printauthor
    \end{flushright}
\end{minipage}
%
\begin{minipage}{0.02\linewidth}
    \rule{1pt}{125pt}
\end{minipage}
\titlepagedecoration
\end{titlepage}
%\lipsum[1-2]
\end{document}